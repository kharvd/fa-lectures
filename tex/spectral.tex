\section{Спектральная теория линейных операторов}
Определение обратного оператора и другие алгебраические аспекты теории можно найти
в <<Лекциях по алгебре>>.

Далее всюду $X$ --- комплексное банахово пространство.

\subsection{Обратные операторы и их свойства}
Пусть $A \colon D(A) \subset X \to X$ --- линейный оператор, определенный на некотором
подпространстве $D(A)$ пространства $X$.

\begin{definition}
    Оператор $A \colon D(A) \subset X \to X$ называется \emph{замкнутым}, если его график
    \[ \Gamma(A) = \menge{(x, Ax) : x \in D(A)} \subset X \times X \]
    является замкнутым подмножеством в пространстве $X \times X$, наделённом нормой
    \[ \norm{(x_1, x_2)} = \max\menge{\norm{x_1}, \norm{x_2}}. \]
\end{definition}

Иначе говоря, оператор замкнут, 
если для всякой сходящейся последовательности $\menge{x_n} \subset D(A)$ такой, что
$Ax_n \to y \in X$, её предел $x$ лежит в $D(A)$ и $y = Ax$.

\begin{example}
    Оператор $A \colon D(A) \subset C[a,b] \to C[a,b]$, $D(A) = C^1[a,b]$, действующий по правилу
    $Ax = x'$, является замкнутым. Это следует из теоремы о почленном дифференцировании
    функциональных последовательностей, известной из курса математического анализа.
\end{example}

\begin{theorem}
    Всякий ограниченный оператор $A \in L(X)$ замкнут.
\end{theorem}

\begin{proof}
    Пусть $A \in L(X)$, $x_n \to x_0$, $Ax_n \to y_0$. В силу непрерывности $A$, $Ax_n \to Ax_0$, значит, в силу единственности предела последовательности, $Ax_0 = y_0$.
\end{proof}

Следующую теорему примем без доказательства. Заметим, что её доказательство опирается
на теорему Бэра.
\begin{theorem}[Банаха о замкнутом графике]\hfill\\
    \indent Пусть $A \colon X \to X$ --- замкнутый линейный оператор, определенный на всем
    банаховом пространстве $X$. Тогда оператор $A$ ограничен.
\end{theorem}

Пусть $A \in L(X)$. Рассмотрим два условия:
\begin{enumerate}
    \item $\ker A = \menge{0}$ --- оператор $A$ инъективен.
    \item $\im A = X$ --- оператор $A$ сюръективен.
\end{enumerate}

В случае, когда $X$ --- конечномерное пространство, как известно из алгебры, эти два условия
эквивалентны. Однако в случае бесконечномерных пространств это не так. Примерами для этого
факта могут служить операторы правого и левого сдвига в $l^\infty$.

Если для оператора из $A \in L(X)$ выполняются условия (1, 2), он является биективным,
а значит существует обратное отображение $A^{-1}$, которое, как известно из алгебры, также
является линейным оператором. Будет ли этот оператор ограниченным? Оказывается, если пространство
$X$ является полным, это всегда так.

\begin{theorem}[Банаха об обратном операторе]\hfill\\
    \indent Пусть линейный оператор $A \in L(X)$, действующий в банаховом пространстве $X$, 
    биективен, т.е. выполнены условия (1) и (2). Тогда $A^{-1}$ ограничен.
\end{theorem}

\begin{proof}
    Поскольку $A$ ограничен, он замкнут. Покажем, что $A^{-1}$ также замкнут.
    \[ \Gamma(A^{-1}) = \menge{(x, A^{-1}x) : x \in X} = \menge{(Ax, x) : x \in X}. \]
    Пусть $Ax_n \to y_0$, а $x_n \to x_0$. Поскольку $A$ замкнут, $y_0 = Ax_0$, и, в силу 
    определения графика,
    $(y_0, x_0) = (Ax_0, x_0) \in \Gamma(A^{-1})$, то есть множество $\Gamma(A^{-1})$ замкнуто.
    Значит, оператор $A^{-1}$ замкнут, а по теореме о замкнутом графике он и ограничен.
\end{proof}

Если $A \colon D(A) \subset X \to X$ определен не на всем пространстве, то для него также можно 
рассматривать условия (1, 2). Тогда будем называть обратным к оператору $A$ оператор 
$A^{-1} \colon X \to X$, который удовлетворяет естественным условиям
\[ AA^{-1} = I_X \]
и
\[ A^{-1}Ax = x \]
для всех $x \in D(A)$. 
Обратим внимание, что мы считаем $A^{-1}$ действующим из $X$ \emph{во всё пространство $X$}, 
а не в $D(A)$.

\begin{theorem}[Банаха об обратном операторе]\hfill\\
    \indent Пусть $A \colon D(A) \subset X \to X$ --- замкнутый биективный линейный оператор, 
    определенный на подмножестве $D(A)$ банахова пространства $X$. Тогда $A^{-1} \colon X \to X$ 
    --- ограниченный оператор. 
\end{theorem}

Доказательство аналогично предыдущему.

\begin{lemma}\label{le:neumann}
    Если $A \in L(X)$ и $\norm{A} < 1$, то оператор $I - A$ обратим, а обратный задается формулой
    \[ (I - A)^{-1} = \sum_{n = 0}^\infty A^n, \]
    причем ряд сходится абсолютно и
    \[ \norm{(I-A)^{-1}} \leq \frac{1}{1-\norm{A}}. \]
\end{lemma}

\begin{proof}
    Покажем, что ряд сходится абсолютно. Используем формулу суммы геометрической прогрессии:
    \[ \sum_{n=0}^\infty \norm{A^n} \leq \sum_{n=0}^\infty \norm{A}^n = \frac{1}{1-\norm{A}}. \]
    Итак, ряд сходится абсолютно, значит он сходится. Отсюда же следует и оценка нормы.
    Обозначим сумму ряда через $B \in L(X)$. Покажем, что $B$ --- обратный к $I - A$.
    \begin{multline*}
        (I - A)B = (I - A)\sum_{n=0}^\infty A^n = \lim_{m\to \infty} (I - A)\sum_{n=0}^m A^n = \\
            = \lim_{m \to \infty} \sum_{n=0}^m (A^n - A^{n+1}) = \lim_{m\to \infty} (I - A^{m+1})
            = I,
    \end{multline*}
    где последнее равенство справедливо в силу условия $\norm{A} < 1$.

    Аналогично доказывается, что $B(I - A) = I$.
\end{proof}

\begin{theorem}
    Пусть $A, B \in L(X)$, $A$ обратим, $\norm{B}\norm{A^{-1}} < 1$. Тогда $A - B$ обратим и
    \[ (A-B)^{-1} = \sum_{n=0}^\infty (A^{-1}B)^n A^{-1}, \]
    и справедлива оценка
    \[ \norm{(A-B)^{-1}} \leq \frac{\norm{A^{-1}}}{1-\norm{B}\norm{A^{-1}}}. \]
\end{theorem}

\begin{proof}
    Предствим оператор $A-B$ в виде $A - B = A(I \hm- A^{-1}B)$. Оператор $A$ обратим, оператор
    $I-A^{-1}B$ обратим в силу леммы. Значит и $A - B$ обратим. Остальное прямо следует из леммы,
    если её применить к оператору $I-A^{-1}B$.
\end{proof}

\subsection{Спектр оператора}
\begin{lemma}
    Если $A \colon D(A) \subset X \to X$ замкнут, то и $A - \lambda I$ замкнут, 
    где $\lambda \in \fieldc$, а $I \colon D(A) \subset X \to X$ --- тождественный оператор.
\end{lemma}

\begin{proof}
    Пусть $A$ замкнут, $\menge{x_n} \subset D(A)$, $x_n \to x$ и $(A \hm- \lambda I)x_n \hm\to y$.
    Тогда
    \begin{multline*}
        \lim_{n\to \infty} Ax_n = \lim_{n\to \infty} (Ax_n - \lambda x_n + \lambda x_n) = 
            \lim_{n\to \infty} (A - \lambda I)x_n + \\ + \lambda \lim_{n\to \infty} x_n 
            = y + \lambda x.
    \end{multline*}
    Тогда, в силу замкнутости $A$,
    \[ Ax = \lambda x + y \Rightarrow (A - \lambda I)x = y, \]
    то есть $A - \lambda I$ также замкнут.
\end{proof}

\begin{definition}
    Пусть $A \colon D(A) \subset X \to X$ --- замкнутый оператор. 
    Будем называть число $\lambda \in \fieldc$ \emph{точкой спектра} оператора $A$, если оператор 
    $ A - \lambda I \colon D(A) \subset X \to X $ необратим, то есть выполнено хотя бы одно из 
    условий
    \begin{enumerate}
        \item $\ker (A - \lambda I) \neq \menge{0}$ --- оператор не инъективен.
        \item $\im (A - \lambda I) \neq X$ --- оператор не сюръективен.
    \end{enumerate}

    Если же число $\lambda \in \fieldc$ не является точкой спектра, то его называют
    \emph{регулярной точкой} оператора $A$.
\end{definition}

Заметим, что по теореме Банаха об обратном операторе, если число $\lambda$ --- регулярная точка
$A$, то оператор $(A - \lambda I)^{-1}$ ограничен.

\begin{definition}
    Множество $\spectrum{A}$ точек спектра оператора $A$ называется \emph{спектром} оператора $A$.
\end{definition}   

\begin{definition}
    Множество $\resset{A} = \fieldc \setminus \spectrum{A}$ регулярных точек оператора $A$
    называется \emph{резольвентным множеством} оператора $A$.
\end{definition}

Спектр оператора принято разбивать на три взаимно непересекающиеся части:
\begin{enumerate}
    \item Дискретный спектр $\spectrum[d]{A}$ --- множество собственных значений оператора $A$, то есть такие $\lambda \in \fieldc$, что $\ker (A - \lambda I) \neq \menge{0}$.
    \item Непрерывный спектр $\spectrum[c]{A}$ --- множество таких $\lambda \in \fieldc$, не
    являющихся собственными значениями, что $\im (A - \lambda I) \neq X$, но 
    $\overline{\im (A - \lambda I)} = X$.
    \item Остаточный спектр $\spectrum[r]{A}$ --- множество точек спектра, не вошедших ни в
    дискретный спектр, ни в непрерывный спектр.
\end{enumerate}

Ясно, что $\spectrum{A} = \spectrum[d]{A} \cup \spectrum[c]{A} \cup \spectrum[r]{A}$.
\begin{definition}
    Отображение $\resolvent{\bullet}{A} \colon \resset{A} \to L(X)$, действующее по правилу
    \[ \resolvent{\lambda}{A} = (A - \lambda I)^{-1}, \]
    называется \emph{резольвентой} оператора $A$.
\end{definition}

\begin{theorem}\label{th:resanalytic}
    Для всякого замкнутого оператора $A$ множество $\resset{A}$ открыто.
    Резольвента $\resolvent{\bullet}{A} \colon \resset{A} \to L(X)$ --- аналитическая функция
    на $\resset{A}$.
\end{theorem}

\begin{proof}
    Пусть $\lambda_0 \in \resset{A}$, а $\lambda \in \fieldc$ таково, что
    \[ \absv{\lambda - \lambda_0} < \frac{1}{\norm{\resolvent{\lambda_0}{A}}}. \]

    Тогда представим оператор $A - \lambda I$ в следующем виде:
    \[ A - \lambda I = A - \lambda_0 I + \lambda_0 I - \lambda I = 
       (A - \lambda_0 I)(I - (\lambda - \lambda_0) \resolvent{\lambda_0}{A}). \]
    Оператор $I - (\lambda - \lambda_0) \resolvent{\lambda_0}{A}$ обратим, поскольку 
    (см. лемму \ref{le:neumann})
    \[ \norm{(\lambda - \lambda_0) \resolvent{\lambda_0}{A})} < 1. \]
    Так как $A - \lambda_0 I$ также обратим, то и $A - \lambda I$ обратим как произведение
    обратимых операторов. Отсюда следует, что резольвентное множество открыто: вместе с каждой
    точкой $\lambda_0$
    в $\resset{A}$ входит открытый круг радиусом меньше $\norm{\resolvent{\lambda_0}{A}}^{-1}$ с 
    центром в точке $\lambda_0$.

    Оператор, обратный к $(I - (\lambda - \lambda_0) \resolvent{\lambda_0}{A})$ представляется
    в виде
    \[ (I - (\lambda - \lambda_0) \resolvent{\lambda_0}{A})^{-1} = 
        \sum_{n=0}^\infty (\lambda - \lambda_0)^n \resolvent{\lambda_0}{A}^n. \]

    Тогда 
    \begin{multline*}
        \resolvent{\lambda}{A} = 
            (A - \lambda I)^{-1} = (I - (\lambda - \lambda_0) \resolvent{\lambda_0}{A})^{-1}
             (A - \lambda_0 I)^{-1} =\\= \sum_{n=0}^\infty (\lambda - \lambda_0)^n 
             \resolvent{\lambda_0}{A}^{n+1}.
    \end{multline*}

    Таким образом мы получили, что $\resolvent{\lambda}{A}$ в некоторой окрестности каждой точки
    $\lambda_0 \in \resset{A}$
    представляется в виде суммы степенного ряда с коэффициентами 
    $c_n = \resolvent{\lambda_0}{A}^{n+1}$. Значит, по теореме \ref{th:tayloranalytic},
    функция $\resolvent{\lambda}{A}$ аналитична на $\resset{A}$.
\end{proof}

\begin{corollaryth}
    Для всякого замкнутого оператора $A$ множество $\spectrum{A}$ замкнуто.
\end{corollaryth}

\begin{theorem}[тождество Гильберта]
    Для любого замкнутого оператора $A$ и любых чисел $\lambda, \mu \in \resset{A}$ 
    справедливо равенство
    \[ \resolvent{\lambda}{A} - \resolvent{\mu}{A} 
        = (\lambda - \mu) \resolvent{\lambda}{A}\resolvent{\mu}{A}. \]
\end{theorem}

\begin{proof}
    Применяя к правой и левой частям равенства $A - \lambda I$ справа и $A - \mu I$ слева, получим
    одинаковые выражения:
    \[ (A - \lambda I)(\resolvent{\lambda}{A} - \resolvent{\mu}{A})(A - \mu I) = 
    A - \mu I - A + \lambda I = (\lambda - \mu) I; \]
    \[ (\lambda - \mu)(A - \lambda I)\resolvent{\lambda}{A}\resolvent{\mu}{A}(A - \mu I) = 
    (\lambda - \mu) I, \]
    то есть
    \begin{multline*}
        (A - \lambda I)(\resolvent{\lambda}{A} - \resolvent{\mu}{A})(A - \mu I) =\\=
          (\lambda - \mu)(A - \lambda I)\resolvent{\lambda}{A}\resolvent{\mu}{A}(A - \mu I).
    \end{multline*} 
    Из биективности $A - \lambda I$ и $A - \mu I$ следует, что на них можно <<сократить>>
    справа и слева. Тогда получаем требуемое равенство.
\end{proof}

\begin{corollaryth}
    Операторы $\resolvent{\lambda}{A}$ и $\resolvent{\mu}{A}$ перестановочны.
\end{corollaryth}

\begin{theorem}[о спектре ограниченного оператора]\label{th:boundedspectrum}\hfill\\
    \indent Пусть $A \in L(X)$ --- ограниченный оператор, действующий в банаховом пространстве $X$.
    Тогда его спектр $\spectrum{A}$ есть непустое компактное множество в $\fieldc$.
\end{theorem}

\begin{proof}
    Сначала покажем, что $\spectrum{A}$ --- компактное множество. Как известно из анализа,
    множество в евклидовом пространстве компактно тогда и только тогда, когда оно замкнуто
    и ограничено. Замкнутость спектра следует из теоремы \ref{th:resanalytic}. Докажем 
    ограниченность.

    Пусть $\absv{\lambda} > \norm{A} \geq 0$. Тогда
    \[ A - \lambda I = -\lambda(I - \lambda^{-1} A). \]
    Оператор $(I - \lambda^{-1} A)$ обратим, поскольку 
    \[ \norm{\lambda^{-1} A} = \frac{\norm{A}}{\absv{\lambda}} < 1. \]
    Тогда и $A - \lambda I$ обратим. Отсюда получаем, что спектр оператора
    $A$ лежит внутри круга радиуса $\norm{A}$ и с центром в нуле, то есть $\spectrum{A}$ --- 
    ограниченное множество и, в силу замкнутости, компактное.

    Покажем, что $\spectrum{A}$ непустое множество. Предположим противное: положим 
    $\resset{A} = \fieldc$ и $\absv{\lambda} > \norm{A}$. Тогда
    при таких $\lambda$ резольвента представляется в виде
    \[ \resolvent{\lambda}{A} = -\sum_{n=0}^\infty \frac{A^n}{\lambda^{n+1}}. \]
    При этом для нормы резольвенты справедлива оценка
    \[
        \norm{\resolvent{\lambda}{A}} \leq 
            \sum_{n=0}^\infty \frac{\norm{A}^n}{\absv{\lambda}^{n+1}} =
            \frac{1}{\absv{\lambda}}\frac{1}{1-\dfrac{\norm{A}}{\absv{\lambda}}}
            = \frac{1}{\absv{\lambda}-\norm{A}} 
            \to 0 \text{ при } \lambda \to \infty.
    \]

    То есть при $\lambda \to \infty$ норма $\norm{\resolvent{\lambda}{A}}$ стремится к нулю.

    При этом, по теореме \ref{th:resanalytic}, резольвента является аналитической функцией на 
    $\resset{A} = \fieldc$, то есть в нашем случае резольвента оказывается целой ограниченной
    функцией (ограниченность следует из стремления к нулю на бесконечности и непрерывности).
    Поэтому, по теореме Лиувилля, $\resolvent{\lambda}{A} = \mathbf{0} \in L(X)$ для всех 
    $\lambda \in \fieldc$, что невозможно. Получили противоречие. Значит спектр оператора $A$ 
    непуст.
\end{proof}

\begin{definition}
    \emph{Спектральным радиусом} линейного ограниченного оператора $A \in L(X)$ называется величина
    \[ r(A) = \max_{\lambda \in \spectrum{A}} \absv{\lambda}. \]
\end{definition}

Спектральный радиус корректно определен в виду компактности спектра $A$ и его непустоты.
Из доказательства теоремы \ref{th:boundedspectrum} видно, что
\[ r(A) \leq \norm{A}, \]
поскольку, если $\absv{\lambda} > \norm{A}$, то оператор $A- \lambda I$ обратим.

\begin{theorem}[формула Бёрлинга-Гельфанда]
    Пусть $A \in L(X)$. Тогда для спектрального радиуса оператора $A$ справедлива формула
    \[ r(A) = \lim_{n\to \infty} \sqrt[n]{\norm{A^n}}. \]
\end{theorem}
