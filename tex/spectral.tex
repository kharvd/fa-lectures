\section{Спектральная теория линейных операторов}
Определение обратного оператора и другие алгебраические аспекты теории можно найти
в <<Лекциях по алгебре>>.

\begin{lemma}
    Если $A \in L(X)$ и $\norm{A} < 1$, то оператор $I - A$ обратим, а обратный задается формулой
    \[ (I - A)^{-1} = \sum_{n = 0}^\infty A^n, \]
    причем ряд сходится абсолютно и
    \[ \norm{(I-A)^{-1}} \leq \frac{1}{1-\norm{A}}. \]
\end{lemma}

\begin{proof}
    Покажем, что ряд сходится абсолютно. Используем формулу суммы геометрической прогрессии:
    \[ \sum_{n=0}^\infty \norm{A^n} \leq \sum_{n=0}^\infty \norm{A}^n = \frac{1}{1-\norm{A}}. \]
    Итак, ряд сходится абсолютно, значит он сходится. Отсюда же следует и оценка нормы.
    Обозначим сумму ряда через $B \in L(X)$. Покажем, что $B$ --- обратный к $I - A$.
    \begin{multline*}
        (I - A)B = (I - A)\sum_{n=0}^\infty A^n = \lim_{m\to \infty} (I - A)\sum_{n=0}^m A^n = \\
            = \lim_{m \to \infty} \sum_{n=0}^m (A^n - A^{n+1}) = \lim_{m\to \infty} (I - A^{m+1})
            = I,
    \end{multline*}
    где последнее равенство справедливо в силу условия $\norm{A} < 1$.

    Аналогично доказывается, что $B(I - A) = I$.
\end{proof}

\begin{theorem}
    Пусть $A, B \in L(X)$, $A$ обратим, $\norm{B}\norm{A^{-1}} < 1$. Тогда $A - B$ обратим и
    \[ (A-B)^{-1} = \sum_{n=0}^\infty (A^{-1}B)^n A^{-1}, \]
    и справедлива оценка
    \[ \norm{(A-B)^{-1}} \leq \frac{\norm{A^{-1}}}{1-\norm{B}\norm{A^{-1}}}. \]
\end{theorem}

\begin{proof}
    Предствим оператор $A-B$ в виде $A - B = A(I-A^{-1}B)$. Оператор $A$ обратим, оператор
    $I-A^{-1}B$ обратим в силу леммы. Значит и $A - B$ обратим. Остальное прямо следует из леммы,
    если её применить к оператору $I-A^{-1}B$.
\end{proof}

Далее $A \colon D(A) \subset X \to X$ --- линейный оператор, определенный на некотором
подпространстве $D(A)$ пространства $X$.

\begin{definition}
    Оператор $A \colon D(A) \subset X \to X$ называется \emph{замкнутым}, если его график
    \[ \Gamma(A) = \menge{(x, Ax) : x \in D(A)} \subset X \times X \]
    является замкнутым подмножеством в пространстве $X \times X$, наделённом нормой
    \[ \norm{(x_1, x_2)} = \max\menge{\norm{x_1}, \norm{x_2}}. \]
\end{definition}

Иначе говоря, оператор замкнут, 
если для всякой сходящейся последовательности $\menge{x_n} \subset D(A)$ такой, что
$Ax_n \to y \in X$, предел $x$ лежит в $D(A)$ и $y = Ax$.

\begin{example}
    Оператор $A \colon D(A) \subset C[a,b] \to C[a,b]$, $D(A) = C^1[a,b]$, действующий по правилу
    $Ax = x'$, является замкнутым. Это следует из теоремы о почленном дифференцировании
    функциональных последовательностей, известной из курса математического анализа.
\end{example}

\begin{theorem}
    Всякий ограниченный оператор замкнут.
\end{theorem}

\begin{proof}
    Пусть $A \in L(X)$, $x_n \to x_0$, $Ax_n \to y_0$. В силу непрерывности $A$, $Ax_n \to Ax_0$, значит, в силу единственности предела последовательности, $Ax_0 = y_0$.
\end{proof}

\begin{theorem}[Банаха о замкнутом графике]
    Пусть \\$A \colon X \to X$ --- замкнутый линейный оператор, определенный на всем
    банаховом пространстве $X$. Тогда оператор $A$ ограничен.
\end{theorem}

\vspace{0.5cm}

Пусть $A \in L(X)$. Рассмотрим два условия:
\begin{enumerate}
    \item $\ker A = \menge{0}$ --- оператор $A$ инъективен.
    \item $\im A = X$ --- оператор $A$ сюръективен.
\end{enumerate}

В случае, когда $X$ --- конечномерное пространство, как известно из алгебры, эти два условия
эквивалентны. Однако в случае бесконечномерных пространств это не так.

Если для оператора из $A \in L(X)$ выполняются условия (1, 2), он является биективным,
а значит существует обратное отображение $A^{-1}$, которое, как известно из алгебры, также
является линейным оператором. Будет ли этот оператор ограниченным? Оказывается, если пространство
$X$ банахово, это всегда так.

\begin{theorem}[Банаха об обратном операторе]
    Пусть линейный оператор $A \in L(X)$, действующий в банаховом пространстве $X$, биективен, т.е. 
    выполнены условия (1) и (2). Тогда $A^{-1}$ ограничен.
\end{theorem}

\begin{proof}
    Поскольку $A$ ограничен, он замкнут. Покажем, что $A^{-1}$ также замкнут.
    \[ \Gamma(A^{-1}) = \menge{(x, A^{-1}x) : x \in X} = \menge{(Ax, x) : x \in X}. \]
    Пусть $Ax_n \to y_0$, а $x_n \to x_0$. Поскольку $A$ замкнут, $y_0 = Ax_0$, и
    $(y_0, x_0) = (Ax_0, x_0) \in \Gamma(A^{-1})$, то есть множество $\Gamma(A^{-1})$ замкнуто.
    Значит, оператор $A^{-1}$ замкнут, а по теореме о замкнутом графике он и ограничен.
\end{proof}
