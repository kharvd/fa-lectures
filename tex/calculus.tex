\section{Элементы функционального исчисления операторов}
\subsection{Операторное исчисление}
Далее $X$ --- комплексное банахово пространство. Обозначим символом
$\mathcal F(\fieldc)$ алгебру целых функций $f \colon \fieldc \to \fieldc$.
Пусть $A \in L(X)$, $f \in \mathcal F(\fieldc)$, а $f$ разлагается в ряд
\[ f(\lambda) = \sum_{n=0}^\infty a_n \lambda^n. \]
Определим отображение $\Phi_A \colon \mathcal F(\fieldc) \to L(X)$ следующим образом:
\[ \Phi_A(f) = f(A) := \sum_{n=0}^\infty a_n A^n. \]
Можно показать, что ряд сходится, а отображение $\Phi_A$ является гомоморфизмом алгебр.

Отображение $\Phi_A$ называется \emph{целым исчислением} оператора $A$.

\begin{example}
    Экспонентой оператора $A \in L(X)$ назовём оператор $e^A$, определяемый формулой
    \[ e^A = \sum_{n=0}^\infty \frac{A^n}{n!}. \]
\end{example}

Рассмотрим более общий вид функционального исчисления операторов.

Обозначим символом $\mathcal F(A)$ множество функций, аналитических на некотором открытом
множестве, содержащем спектр $\spectrum{A}$ оператора $A \in L(X)$. Это множество является алгеброй с
поточечными операциями сложения и умножения: если $f \colon U_1 \supset \spectrum{A} \to \fieldc$,
$g \colon U_2 \supset \spectrum{A} \to \fieldc$, то $f + g$ и $fg$ действуют из $U_1 \cap U_2 \supset \spectrum{A}$ в
$\fieldc$ по правилу

\[ \begin{aligned}
    (f + g)(z) &= f(z) + g(z), \\
    (fg)(z) &= f(z)g(z)
\end{aligned} \quad\quad z \in U_1 \cap U_2. \]

Вспомним интегральную формулу Коши:
\[ f(z) = -\frac{1}{2\pi i} \int_{\gamma} \frac{f(\lambda)}{z - \lambda} \dd \lambda. \]

Идея \emph{исчисления Данфорда} (еще говорят \emph{голоморфного функционального исчисления},
\emph{операторного исчисления}) состоит в том, чтобы использовать интегральную формулу Коши для 
определения значения функции от оператора.

Пусть $A \in L(X)$. Определим отображение $\Psi_A \colon \mathcal F(A) \to L(X)$ по правилу
\[ \Psi_A(f) = f(A) 
    := -\frac{1}{2\pi i} \int_{\gamma} f(\lambda) \resolvent{\lambda}{A} \dd \lambda, \]
где контур $\gamma$ --- граница открытого множества $V \supset \spectrum{A}$, лежащего в множестве
аналитичности функции $f$.

Отображение $\Psi_A$ называется \emph{исчислением Данфорда} оператора $A$ или просто
\emph{операторным исчислением}. Следующая теорема обосновывает корректность такого названия.

\begin{theorem}\hfill\\
    \indent Отображение $\Psi_A$ является гомоморфизмом алгебры $\mathcal F(A)$ в алгебру $L(X)$, то есть для всех
    $f, g \in \mathcal F(A)$ справедливо
    \[ (f+g)(A) = f(A) + g(A), \]
    \[ (fg)(A) = f(A)g(A). \]

    Кроме того, если $f$ --- целая функция, то
    \[ f(A) = \sum_{n=0}^\infty a_n A^n, \]
    то есть целое исчисление и исчисление Данфорда совпадают для целых функций.
\end{theorem}

\begin{proof}
    Первое свойство следует из линейности интеграла по контуру. Докажем второе свойство.    
    Пусть $U_1$ и $U_2$ --- открытые множества, содержащие спектр, причем такие, что замыкание 
    $U_1$ лежит в $U_2$, а замыкание $U_2$ лежит в общем множестве аналитичности функций $f$ и $g$.
    Символами $\gamma_1$ и $\gamma_2$ обозначим контуры, обходящие границы $U_1$ и $U_2$ 
    соответственно в положительном направлении обхода (так, чтобы внутренность множества 
    оставалась слева). Тогда, применяя интегральную формулу Коши и тождество Гильберта, получим
    \begin{align*}
        f(A)&g(A) = \frac{1}{4 \pi^2}\int_{\gamma_1} f(\lambda) \resolvent{\lambda}{A} \dd \lambda
        \cdot 
        \int_{\gamma_2} f(\mu) \resolvent{\mu}{A} \dd \mu = \\[0.7em]
        &= \frac{1}{4\pi^2} \int_{\gamma_1} \int_{\gamma_2} f(\lambda) g(\mu) 
        \resolvent{\lambda}{A} \resolvent{\mu}{A} \dd \mu \dd \lambda = \\[0.7em]
        &= \frac{1}{4\pi^2} \int_{\gamma_1} \int_{\gamma_2} f(\lambda) g(\mu) (\lambda - \mu)^{-1}
        (\resolvent{\lambda}{A} - \resolvent{\mu}{A}) \dd \mu \dd \lambda = \\[0.7em]
        &= \frac{1}{4\pi^2} \int_{\gamma_1} \int_{\gamma_2} f(\lambda) g(\mu) (\lambda - \mu)^{-1}
        \resolvent{\lambda}{A} \dd \mu \dd \lambda - \\[0.7em]
        &\quad- \frac{1}{4\pi^2} \int_{\gamma_1} \int_{\gamma_2} f(\lambda) g(\mu) (\lambda - \mu)^{-1}
        \resolvent{\mu}{A} \dd \mu \dd \lambda = \\[0.7em]
        &= -\frac{1}{2\pi i} \int_{\gamma_1} f(\lambda) \resolvent{\lambda}{A} \left( 
        -\frac{1}{2\pi i} \int_{\gamma_2} \frac{g(\mu)}{\lambda - \mu} \dd \mu \right) 
        \dd \lambda - \\[0.7em]
        &\quad- \left(-\frac{1}{2\pi i} \right) \int_{\gamma_2} g(\mu) \resolvent{\mu}{A} 
        \left( -\frac{1}{2\pi i} \int_{\gamma_1} \frac{f(\lambda)}{\lambda - \mu} \dd \lambda \right) 
        \dd \mu = \\[0.7em]
        &= -\frac{1}{2\pi i} \int_{\gamma_1} f(\lambda) \resolvent{\lambda}{A} \left( 
        -\frac{1}{2\pi i} \int_{\gamma_2} \frac{g(\mu)}{\lambda - \mu} \dd \mu \right) 
        \dd \lambda = \\[0.7em]
        &= -\frac{1}{2\pi i} \int_{\gamma_1} f(\lambda) g(\lambda) \resolvent{\lambda}{A}
        \dd \lambda = (fg)(A),
    \end{align*}
    где интеграл
    \[ \int_{\gamma_1} \frac{f(\lambda)}{\lambda - \mu} \dd \lambda \]
    равен нулю, поскольку $\mu$ лежит за пределами $U_1$ (на контуре $\gamma_2$), то есть функция
    \[ h(\lambda) = \frac{f(\lambda)}{\lambda - \mu} \]
    аналитична в области $U_1$ (знаменатель в ноль не обращается).

    Третье свойство дано без доказательства.
\end{proof}

\begin{theorem}[Данфорда об отображении спектра]\hfill\\
    \indent Пусть $A \in L(X)$, $f \in \mathcal F(A)$. Тогда
    \[ \spectrum{f(A)} = f(\spectrum{A}) = \menge{f(\lambda) : \lambda \in \spectrum{A}}. \]
\end{theorem}

\subsection{Проекторы Рисса}
\begin{theorem}
Пусть спектр оператора $A \in L(X)$ представим в виде объединения двух непересекающихся
замкнутых частей: $\spectrum{A} = \sigma_1 \cup \sigma_2$. Тогда существует разложение $X$
в прямую сумму замкнутых подпространств $X = X_1 \oplus X_2$, причем пространства $X_1$ и $X_2$ 
инвариантны относительно оператора $A$. Более того, если $A_k = A|_{X_k}$, $k=1,2$, то $\spectrum{A_1} = \sigma_1$ и
$\spectrum{A_2} = \sigma_2$.
\end{theorem}

\begin{proof}
    Определим функцию $f \colon U_1 \cup U_2 \to \fieldc$ по правилу
    \[ f(\lambda) = 
        \begin{cases}
            1, & \lambda \in U_1, \\
            0, & \lambda \in U_2,
        \end{cases} \]
    где $U_1$ и $U_2$ --- взаимно непересекающиеся открытые множества, содержащие $\sigma_1$ и 
    $\sigma_2$ соответственно. Очевидно, что $f$ --- аналитическая функция: она дифференцируема
    в каждой точке $U_1 \cup U_2$, то есть $f \in \mathcal F(A)$.
    Значит можно определить оператор $f(A)$:
    \begin{align*}
        f(A) &= -\frac{1}{2\pi i} \int_{\gamma} f(\lambda) \resolvent{\lambda}{A} \dd \lambda 
            = -\frac{1}{2\pi i} \int_{\gamma_1} f(\lambda) \resolvent{\lambda}{A} \dd \lambda - \\ 
            &-\frac{1}{2\pi i} \int_{\gamma_2} f(\lambda) \resolvent{\lambda}{A} \dd \lambda =
            -\frac{1}{2\pi i} \int_{\gamma_1} \resolvent{\lambda}{A} \dd \lambda,
    \end{align*}
    где $\gamma$ --- граница $U_1 \cup U_2$, являющаяся объединением $\gamma_k$ --- границ $U_k$, 
    $k=1,2$.

    Введем обозначение $P_1 = f(A)$. Покажем, что $P_1$ --- проектор. Поскольку 
    $(f\cdot f)(\lambda) = (f(\lambda))^2 = f(\lambda)$ для всех $\lambda \in U_1 \cup U_2$, 
    в силу определения гомоморфизма алгебр, получаем:
    \[ P_1^2 = f(A)f(A) = (f \cdot f)(A) = f(A) = P_1. \]
    Итак, $P_1$ в самом деле проектор.

    Пусть $X_1 = \im P_1$, $X_2 = \ker P_1$. Из алгебры известно, что пространство $X$
    раскладывается в прямую сумму $X_1$ и $X_2$. Покажем, что пространства $X_1$ и $X_2$ 
    инвариантны относительно $A$. Для этого достаточно показать, что $AP_1 = P_1A$ (см. алгебру).
    \begin{align*}
        AP_1 &= -\frac{1}{2\pi i} \int_{\gamma_1} A \resolvent{\lambda}{A} \dd \lambda, \\
        P_1A &= -\frac{1}{2\pi i} \int_{\gamma_1} \resolvent{\lambda}{A} A \dd \lambda.
    \end{align*}

    Легко показать, что для любого оператора $A \in L(X)$ и любого $\lambda \in \resset{A}$ 
    справедливо равенство\footnote{Рассмотрите очевидное равенство $(A- \lambda I)A = 
    A(A- \lambda I)$}
    \[ A\resolvent{\lambda}{A} = \resolvent{\lambda}{A}A.\]
    Отсюда получаем, что в самом деле пространства $X_1$ и $X_2$ инвариантны относительно $A$.
    Значит можно определить сужения $A|_{X_1} = A_1 \in L(X_1)$, $A|_{X_2} = A_2 \in L(X_2)$.
    Утверждение о спектре этих сужений оставим без доказательства.
\end{proof}
