\section[Принцип равномерной ограниченности]{Принцип равномерной ограниченности
(теорема Банаха-Штейнгауза)}

\begin{definition}
    Множество из метрического пространства называется \emph{множеством I
    категории
    (<<тощим>>, разреженным)}, если его можно представить в виде счетного объединения замкнутых
    множеств, каждое из которых не содержит шара.
\end{definition}

\begin{definition}
    Множество, не являющееся множеством I категории, называется \emph{множеством
    II категории (<<тучным>>)}.
\end{definition}

\begin{theorem}[Бэра]
    Всякое полное метрическое пространство является множеством II категории.
\end{theorem}

Пусть $X$ и $Y$ --- банаховы пространства, $\Omega$ --- множество индексов,
$\menge{A_\alpha}_{\alpha \in \Omega}$ --- семейство ограниченных операторов.

Будем называть семейство операторов \emph{ограниченным поточечно}, если для
каждого $x \in X$ существует такая константа $M(x) > 0$, что
\[ \norm{A_\alpha x} \leq M(x) \]
для всех $\alpha \in \Omega$, то есть для каждого $x \in X$ множество
\[ \menge{A_\alpha x : \alpha \in \Omega} \subset Y \]
ограничено в $Y$.

Семейство операторов назовём \emph{ограниченным равномерно}, если существует такое
число $C > 0$, что для всех $\alpha \in \Omega$ выполнено неравенство
\[ \norm{A_\alpha} < C, \]
то есть числовое множество
\[ \menge{\norm{A_\alpha} : \alpha \in \Omega} \]
ограничено.

\begin{theorem}[Банаха-Штейнгауза]
    Если семейство ограниченных операторов $\menge{A_\alpha}_{\alpha \in \Omega}$,
    действующих из банахова пространства $X$ в нормированное пространство $Y$, 
    ограничено поточечно, то оно ограничено и равномерно. 
\end{theorem}

\begin{proof}
    Рассмотрим множества вида
    \[ X_n = \menge{x \in X : \forall \alpha \in \Omega \; \norm{A_\alpha x} \leq n}. \]
    В силу поточечной ограниченности семейства, $X = \bigcup\limits_{n=1}^\infty X_n$.

    Каждое из множеств $X_n$ замкнуто. В самом деле: если $\menge{x_k}$ ---
    сходящаяся к $x_0 \in X$ последовательность элементов из $X_n$, то, в силу непрерывности
    операторов $A_\alpha$, $\lim\limits_{k\to\infty}\norm{A_\alpha x_k}
    =\norm{A_\alpha x_0}$, а поскольку для всех $x_k$ и всех $\alpha \in \Omega$
    выполняется неравенство $\norm{A_\alpha x_k} \leq n$, то и $\norm{A_\alpha
    x_0} \leq n$, а значит $x_0 \in X_n$, что и означает замкнутость $X_n$.

    Поскольку пространство $X$ полно, по теореме Бэра существует такой номер
    $n_0$, что $X_{n_0}$ содержит в себе шар, который будем 
    обозначать $B(x', r)$, где $r$ --- радиус этого шара, а
    $x'$ --- его центр.

    Для всех элементов $x$ из $B(x', r)$ и для всех $\alpha \in \Omega$ справедливо, что
    \[ \norm{A_\alpha x} \leq n_0, \]
    то есть значения $\norm{A_\alpha x}$ ограничены на этом шаре. Покажем, что они
    ограничены и на единичном шаре, что будет означать ограниченность норм
    $A_\alpha$. 
    
    Пусть $x \in B(0, 1)$. Тогда, как нетрудно
    проверить, $z = rx + x' \in B(x', r)$. В таком случае для всех $\alpha \in
    \Omega$
    \[ \norm{A_{\alpha} x} = \norm{A_{\alpha} \left( \frac{z - x'}{r}
    \right)}\leq \frac{1}{r} (\norm{A_\alpha z} + \norm{A_\alpha x'}) \leq
    \frac{2n_0}{r}, \]
    откуда, взяв верхнюю грань по всем $x \in B(0, 1)$, получаем утверждение
    теоремы.
\end{proof}
